\documentclass[a4j, 12pt,leqno]{jsarticle}
\input{preamble2.tex}
\usepackage{bussproofs}
\usepackage{amssymb}
\usepackage{okumacro}
\usepackage{plext}
\usepackage{latexsym}
\usepackage{lscape}
\usepackage{stackengine} 




% This is the "centered" symbol
\def\fCenter{{\mbox{\Large$\rightarrow$}}}

% Optional to turn on the short abbreviations
\EnableBpAbbreviations

% \alwaysRootAtTop  % makes proofs upside down
% \alwaysRootAtBottom % -- this is the default setting
\title{カント読書会議事録\thanks{ページづけはPhB版(1998)による。AはAbsatz(段落)の略記。内容は常に不完全にとどまる。ご連絡は以下のアドレスまでお願いいたします。}}
\author{文責:福地信哉\thanks{e-mail: shinya.express@gmail.com}\\ (東京大学大学院人文社会系研究科修士課程)}


\begin{document}
\maketitle


\section*{論点}
\begin{description}
\item[S. 111, A. 2]現象と物自体の区別について。解釈。この区別の正当化としては例えば次の道筋が考えられる。まず、世界は私たちによって知られうる事柄と知られえない事柄から成っているという無害な前提を置く。次に、私たちによって知られうる以上、知られうることが私たちの能力の性状から原理的に決定しており、他方知られえない以上、知られえないことが私たちの能力によって原理的に決定しているというステップを踏む。そして前者の事柄は現象に、後者の事柄は物自体に帰せられる。
\begin{prooftree}
\AxiomC{$\Diamond kP \lor \lnot \Diamond kP$}
\AxiomC{[$\Diamond kP$]$^1$}
\UnaryInfC{$g\Diamond kP $}
\UnaryInfC{$g\Diamond kP \lor g\lnot \Diamond kP $}
\AxiomC{[$\lnot\Diamond kP$]$^2$}
\UnaryInfC{$g\lnot\Diamond kP$}
\UnaryInfC{$g\Diamond kP \lor g\lnot \Diamond kP $}
\RightLabel{\scriptsize 1, 2}
\TrinaryInfC{$g\Diamond kP \lor g\lnot \Diamond kP $}
\end{prooftree}
\item[・]診断。「次に」のステップは問題含み。

\end{description}
\begin{description}
\item[S. 112, Z. 21ff.]解釈。カントはここでオブジェクトoが超越論的に観念的である(以下TI)ことの一つの基準を、oの存在を認めないタイプの認識者が思考可能であることとし、また時間がそのようなオブジェクトであると主張しているように見える。そうだとして他方、oが超越論的に実在的である(以下TR)ことの基準は何か。自然に考えられるのは、思考可能なあらゆるタイプの認識者によってoが存在するとみなされているということだ。
\item[・]診断。だがTRのこうした特徴づけは修正を要する。第一に、具体的にどのようなタイプの認識者までが考慮されてoがTRないしTIであると判定されるのかが不明瞭である(思考可能性の問題)。
\item[・]第二にやや軽微な問題。思考可能なあらゆるタイプの認識者によって存在するとみなされているもの(TRなもの)はそもそもただ一つでも存在するのか。もし存在しないとすれば、TIなものをTRなものから区別することの有意味性が疑われることになる。
\begin{description}
\item[・]とは言えその場合でも、「そのような存在しないものを存在すると人々が常識的に信じている(例えば時空について)」という主張としてカントの主張を理解することは依然可能かもしれない。
\end{description}
\item[・]そして第三に重要な問題。あらゆるタイプの認識者によって存在するとみなされているオブジェクトのみをTRなものとすると次のことが帰結する。それは世界を捉える能力が極めて高い認識者によって存在すると見なされているoが、その能力の極めて低い認識者がoを存在するとみなさないというだけで、TRでないと結論されてしまうことである。
\end{description}
\begin{description}
\item[S. 113, A. 2]要約。敵によれば心的状態(表象状態)の現実性は確実である一方、外的対象(物自体)のそれは不確実である。Kの立場ではこのような非対称性はなく、心的状態も外的対象も同等に現実的であり、表象/物自体の対立は確実/不確実の対比に呼応しない。ここは、表象/物自体が2種類の存在者ではなく同じ存在者の2側面であることが強調されている箇所。
\item[・] 問題。敵が時間の現実性を主張する一方空間のそれを主張しないのは、観念論に直面するため。敵はTRを採用しているが、TRに対して観念論が直撃する所以は何か?
\begin{description}
\item[・]答え。第4誤謬推理によれば観念論は、1.真正の経験系列と幻覚経験の系列が識別不可能であり前者の系列において私たちは外的対象の存在を知っているが後者の系列においては知らない、2.対象の存在を知るルートは経験系列の性質からの推論しかない、という2前提から帰結する。TRが効いているのは第1前提である。彼らによれば、その存在が知られるべき外的対象とは、経験系列に対応するものなのである。他方 TIにおいてその存在が知られるべき外的対象とは経験系列に内部化された存在者である。
\item[・]問題。TRなものを、経験系列に対応する存在者として特徴付けるのと、すべての(あるいは神的な)認識者によって認識されるものとして特徴付けるのはどのように整合するのか?
\begin{description}
\item[・]答え。このような認識論的な特徴づけと存在論的な特徴づけがカントにあって癒着しているのは、神的な認識の対象と私たちのような受容的なタイプの認識者から独立に存在する対象とが外延的に同一であるためだろう。
\end{description}
\end{description}
\item[S. 114, Z. 1--20]TIが幾何学の空間に対するアプリオリな適用を保証し、またその適用範囲を制限するということの2点が述べられている。双方ともTIの強い主張(空間は私たちの感性の条件でしかない)に依っている。私たちの感性の条件でしかないからこそ、もしかしたら適用されていないかもしれないという疑いが除去され、また適用範囲が現象に限定されるのだ。
\item[・]
カントはここで、TIとTRという哲学的対立が数学的認識の確実性について異なった帰結をもつのに対して、経験的認識のそれをめぐっては異なった帰結をもたないと述べている(Z. 15ff.)これは第4誤謬推理などで、TIだからこそ経験的認識の確実性が保証されると述べていることと平仄が合わないが(TIは経験的認識の確実性を実際は引き上げるのだがここでは引き下げはしないという述べるにとどめている) 、ここではあくまでも数学的認識の確実性が主題であるがゆえにこのように省略的な表現をしていると理解できる。
\end{description}

\end{document}