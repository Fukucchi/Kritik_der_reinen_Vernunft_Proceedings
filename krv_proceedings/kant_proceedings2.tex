\documentclass[12pt]{jsarticle}
\usepackage{okumacro}
\usepackage{plext}
\usepackage{amsmath}
\usepackage[stable]{footmisc}
\usepackage{mathtools}
\usepackage{geometry}
\usepackage{breqn}
\usepackage[dvipdfmx]{graphicx}
\usepackage{here}
\setcounter{section}{0}
\geometry{top=25mm,bottom=25mm,left=25mm,right=25mm}
\usepackage{bussproofs}
\usepackage{amssymb}
\usepackage{okumacro}
\usepackage{plext}
\usepackage{latexsym}
\usepackage{lscape}
\usepackage{stackengine} 




% This is the "centered" symbol
\def\fCenter{{\mbox{\Large$\rightarrow$}}}

% Optional to turn on the short abbreviations
\EnableBpAbbreviations

% \alwaysRootAtTop  % makes proofs upside down
% \alwaysRootAtBottom % -- this is the default setting
\title{カント読書会議事録\thanks{ページづけはPhB版(1998)による。AはAbsatz(段落)の略記。内容は常に不完全にとどまる。ご連絡は以下のアドレスまでお願いいたします。}}
\author{文責:福地信哉\thanks{e-mail: shinya.express@gmail.com}\\ (東京大学大学院人文社会系研究科修士課程)}


\begin{document}
\maketitle


\section{2020/3までの読書会での論点}
\begin{description}
\item[S. 111, A. 2]現象と物自体の区別について。解釈。この区別の正当化としては例えば次の道筋が考えられる。まず、世界は私たちによって知られうる事柄と知られえない事柄から成っているという無害な前提を置く。次に、私たちによって知られうる以上、知られうることが私たちの能力の性状から原理的に決定しており、他方知られえない以上、知られえないことが私たちの能力によって原理的に決定しているというステップを踏む。そして前者の事柄は現象に、後者の事柄は物自体に帰せられる。
\begin{prooftree}
    \AxiomC{$\Diamond kP \lor \lnot \Diamond kP$}
    \AxiomC{[$\Diamond kP$]$^1$}
    \UnaryInfC{$g\Diamond kP $}
    \UnaryInfC{$g\Diamond kP \lor g\lnot \Diamond kP $}
    \AxiomC{[$\lnot\Diamond kP$]$^2$}
    \UnaryInfC{$g\lnot\Diamond kP$}
    \UnaryInfC{$g\Diamond kP \lor g\lnot \Diamond kP $}
    \RightLabel{1, 2}
    \TrinaryInfC{$g\Diamond kP \lor g\lnot \Diamond kP $}
\end{prooftree}
\item[・]診断。「次に」のステップは問題含み。

\end{description}
\begin{description}
\item[S. 112, Z. 21ff.]解釈。カントはここでオブジェクトoが超越論的に観念的である(以下TI)ことの一つの基準を、oの存在を認めないタイプの認識者が思考可能であることとし、また時間がそのようなオブジェクトであると主張しているように見える。そうだとして他方、oが超越論的に実在的である(以下TR)ことの基準は何か。自然に考えられるのは、思考可能なあらゆるタイプの認識者によってoが存在するとみなされているということだ。
\item[・]診断。だがTRのこうした特徴づけは修正を要する。第一に、具体的にどのようなタイプの認識者までが考慮されてoがTRないしTIであると判定されるのかが不明瞭である(思考可能性の問題)。
\item[・]第二にやや軽微な問題。思考可能なあらゆるタイプの認識者によって存在するとみなされているもの(TRなもの)はそもそもただ一つでも存在するのか。もし存在しないとすれば、TIなものをTRなものから区別することの有意味性が疑われることになる。
\begin{description}
\item[・]とは言えその場合でも、「そのような存在しないものを存在すると人々が常識的に信じている(例えば時空について)」という主張としてカントの主張を理解することは依然可能かもしれない。
\end{description}
\item[・]そして第三に重要な問題。あらゆるタイプの認識者によって存在するとみなされているオブジェクトのみをTRなものとすると次のことが帰結する。それは世界を捉える能力が極めて高い認識者によって存在すると見なされているoが、その能力の極めて低い認識者がoを存在するとみなさないというだけで、TRでないと結論されてしまうことである。
\end{description}
\begin{description}
\item[S. 113, A. 2]要約。敵によれば心的状態(表象状態)の現実性は確実である一方、外的対象(物自体)のそれは不確実である。Kの立場ではこのような非対称性はなく、心的状態も外的対象も同等に現実的であり、表象/物自体の対立は確実/不確実の対比に呼応しない。ここは、表象/物自体が2種類の存在者ではなく同じ存在者の2側面であることが強調されている箇所。
\item[・] 問題。敵が時間の現実性を主張する一方空間のそれを主張しないのは、観念論に直面するため。敵はTRを採用しているが、TRに対して観念論が直撃する所以は何か?
\begin{description}
\item[・]答え。第4誤謬推理によれば観念論は、1.真正の経験系列と幻覚経験の系列が識別不可能であり前者の系列において私たちは外的対象の存在を知っているが後者の系列においては知らない、2.対象の存在を知るルートは経験系列の性質からの推論しかない、という2前提から帰結する。TRが効いているのは第1前提である。彼らによれば、その存在が知られるべき外的対象とは、経験系列に対応するものなのである。他方 TIにおいてその存在が知られるべき外的対象とは経験系列に内部化された存在者である。
\item[・]問題。TRなものを、経験系列に対応する存在者として特徴付けるのと、すべての(あるいは神的な)認識者によって認識されるものとして特徴付けるのはどのように整合するのか?
\begin{description}
\item[・]答え。このような認識論的な特徴づけと存在論的な特徴づけがカントにあって癒着しているのは、神的な認識の対象と私たちのような受容的なタイプの認識者から独立に存在する対象とが外延的に同一であるためだろう。
\end{description}
\end{description}
\item[S. 114, Z. 1--20]TIが幾何学の空間に対するアプリオリな適用を保証し、またその適用範囲を制限するということの2点が述べられている。双方ともTIの強い主張(空間は私たちの感性の条件でしかない)に依っている。私たちの感性の条件でしかないからこそ、もしかしたら適用されていないかもしれないという疑いが除去され、また適用範囲が現象に限定されるのだ。
\item[・]
カントはここで、TIとTRという哲学的対立が数学的認識の確実性について異なった帰結をもつのに対して、経験的認識のそれをめぐっては異なった帰結をもたないと述べている(Z. 15ff.)これは第4誤謬推理などで、TIだからこそ経験的認識の確実性が保証されると述べていることと平仄が合わないが(TIは経験的認識の確実性を実際は引き上げるのだがここでは引き下げはしないという述べるにとどめている) 、ここではあくまでも数学的認識の確実性が主題であるがゆえにこのように省略的な表現をしていると理解できる。
\end{description}
\section{S. 114, Z. 20 - S. 115, Z. 27 (2020/3/24)}
カントは空間の性質のアプリオリ性と空間が限定的領域のみに妥当することの二点を説明するという点で自らの強みを打ち出している。そして、ニュートンの立場では前者はOKだが後者はNG、ライプニッツの立場では前者はNGだが後者はOK、自分の立場だとどっちもOKという診断を下している。

第一の問題はそもそも空間の性質のアプリオリ性を説明できるオプションにスコアが入るという考えの妥当性である。この点についてはここで立ち入らない。

第二のここで注目したい問題は、カントは〈ライプニッツは、空間が限定的領域にのみ妥当するとわかっていた、言い換えると現象/物自体の区別が出来ていた(S. 115, Z. 15ff.)〉と褒めているが、それは誤解を導くということである。正確に言えば、カント、ニュートン、ライプニッツの立場は次のような二つの基準によって分類される。
\begin{itemize}
    \item 認識者を消した時に空間的構造が宇宙に残存するか?
    \begin{itemize}
        \item no:カント
        \item yes:関係項を除去した時に空間的構造が宇宙に残存するか?
        \begin{itemize}
            \item no:ライプニッツ
            \item yes:ニュートン
            
        \end{itemize}

            
        
    \end{itemize}
    
\end{itemize}
第1の問にyes と答えると第1の行き過ぎを犯すことになる。ここにあるのは空間を物自体へ拡張するなという規則である。第2の問いにyes と答えると第2の行き過ぎを犯すことになる。これが関係(順序対の集合)を関係項なしに存立するものとして誤解するなという規則である。

つまりライプニッツの褒められるべきは第2の行き過ぎを犯していないという点である(カントの書き振りだと第1の行き過ぎを犯していないという点で褒められているように見えるものの)。

\section{S. 115, Z. 28 - S. 116, Z. 9 (2020/4/22)}

まずはテクストを離れて感性論全体の問題点を確認した。
「認識者を消しても宇宙に時空的構造が保たれる」という命題を$P$とする。
カントが支持しているのは、$\lnot kP$、
$k\lnot P$、そして$\lnot P$である。
\begin{itemize}
    \item 第1の解釈問題は、この間の移行がどのような前提を入れてなされるのかということだ。
    \item 第2のそれに先立つ解釈問題は、$P$の意味である。
    \begin{itemize}
        \item まず認識者を消すということに二義性がある。
        \begin{itemize}
            \item 第1に、単に認識者が死ぬということである。しかしそうだとするとカントの主張している$\lnot P$は明らかにもっともらしくない。
            1人死んだところで時空はあるのに
            全員死ぬとなくなることになるからだ。
            \item 第2に、認識者が彼の枠組みを度外視して考えるということである。そうすると2つの問題が生じる。
            \begin{itemize}
                \item $\lnot P$は、「認識者が想像の規則を度外視して世界を考察するときには、世界は想像の規則を度外視して考察されている」というトートロジーになる。
                \item 「認識者が想像の規則を度外視して世界を考察する」ことの内実が不明瞭。
        
            \end{itemize}
        \end{itemize}
        \item 次に、「宇宙に時空的構造が保たれる/消える」ということの真理条件が不明瞭。
    \end{itemize}
\end{itemize}

続いて、S. 115, Z. 28 - S. 116, Z. 9の段落を検討した。
この段落では超越論的感性論の考察の対象が時空に限定されており運動(物体の位置変化)や変化(
物体の状態変化)が含まれない理由が述べられている。理由は、
運動や変化の概念は経験的だからというものだ。

これは次のように解釈される。
一方で運動や変化の概念は物体を抜きにすると
定義上その適用先を失う。
他方で空間や時間の概念は物体を抜きにしてもその適用先を失わない。
ここで適用先を失わないとは、
そのような適用先が物理的に実在しうるということまでは意味しない。
それゆえ、これによって絶対空間説に与することはない。

\section{S. 116, Z. 10 -- S. 118, Z. 17 (2020/5/10)}

S. 116, Z. 17 -- S. 117, Z. 22の段落の前半部(- S. 116, Z.27)は,物自体は時空的ではない($\lnot p$)と述べている.後半部は物自体についての命題が不可知である(それゆえ$\lnot k p \land \lnot k \lnot p$)と述べている.

まずい点.まず $\lnot k \lnot p$についていえば,これは直前で主張していたことを知っていないと述べることに相当する.次に$\lnot k p$について.第1に,これが$\lnot p$からの帰結であるとしよう.そうだとすると$\lnot p$という強い主張は前提なしに入れられていることになる.第2に,$\lnot p$を出すための前提であるとしよう.$\lnot k p$とは大雑把に言って,$\lnot p \lor p\ is\ not\ justified$なので,ここから$\lnot  p$を導出するには,$p$という信念が正当化されていると言う必要がある.それは物自体が時空的だという信念が正当化されているということであり,これは文意に沿わない.

S. 117, Z. 23 -- S. 118, Z. 17の段落では,時空的に世界を表象する/そうでない理想的な仕方で世界を表象するとは,同じ対象を違う解像度で表象することではなく,むしろ現象と物自体という別の対象を表象することだと主張されている.この2つの立場は共に,両表象様式の間に越えられない壁があるということは認めている.異なるのは両者の間に上下関係を設定するか否かということであり,これが表象間に程度差しか認めないか種別を設けるかという違いの内実である.なお,上下関係を設定することへの反論はここではなされていない.立場の相違が明確にされているだけである.

\end{document}